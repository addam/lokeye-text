\chapwithtoc{Attachments}

\setcounter{section}{0}
\stepcounter{chapter}
\renewcommand{\thesection}{\Alph{section}}

\section{Directory structure}
\label{s:dirstructure}

The attached data is organized as follows:

\begin{itemize}
\item The directory {\tt bin/} contains a pre-built binary for Linux x86\_64 systems with glibc 2.4.

\item {\tt readme.html} are brief instructions for installation and usage.

\item The directory {\tt src/} contains all of the the source code of the program, in version from 2017-07.

\item The directory {\tt screencapture/} is a MSVC++ project helper tool for acquisition of ground truth data from the Tobii EyeX tracker \cite{tobii}.
It has been developed as a part of this thesis, and we believe that it may be helpful to other users of the EyeX tracker.
Note that this program can be only ran on Windows since the Tobii API is also limited to that platform.

\item The directory {\tt opencv/} contains a distribution of the OpenCV 3.1 library, which is necessary for building our program.
Instead of building this library from source, the user is recommended to download a release specific to their system from \href{http://opencv.org}.

\item The directory {\tt data/} contains the testing data sets, as described in the following attachment.

\item Finally, the file {\tt thesis.pdf} is the electronic version of this document.

\end{itemize}

\section{Testing Data}
\label{s:testingdata}

There are three data sets provided in the testing data directory.
All of them have been manually annotated with the position of iris center, and with a few extra values that describe the qualitative aspects of each image.

\begin{itemize}
\item {\tt data/instagram/} are pairs of eyes acquired from random photos tagged as ``selfie" on the website \ahref{https://www.instagram.com/}.
The users who took the original photos come from all around the globe.
By the nature of this data source, there are much more female subjects than males, and subjectively it seems that the ethnicities are not distributed accordingly to their global distribution.
However, the images have been indeed selected by a random search in a fair manner.

\item {\tt data/models/} pairs of eyes cropped out from photos of photo models, taken by professionals.
The visual quality of most of the images is excellent, being perfectly in focus and providing enough contrast.
All of the subjects are females, and most of them wear makeup.
These images have been obtained from many different, publicly available sources.

\item {\tt data/eyes/} are single eyes cropped out from the personal archive of the author.
The single exception is the image {\tt eyes/lena.jpg}, which is cropped out from the profound Lena image.
The situations in which each of the photos were acquired vary greatly, and so does their visual quality and resolution.
\end{itemize}

The annotations are supplied in the file {\tt data/annotations.js} in JSON format, and have been evaluated subjectively by the author of this thesis.
For precise localization of the iris center, a specific tool was developed that is also provided in the source code.

There are also three videos with ground truth data that can be used for an offline evaluation of the program (without using a webcamera).
The iris centers ($x, y$ coordinates in pixel units) are listed in CSV format, in the order of the corresponding video frames.
Each of these {\tt .avi} videos is accompanied by a {\tt .csv} annotation.
The people in these videos have been asked to do some basic tasks on the computer, in order for this data to be realistic.
