\chapter{Goals}

\todo{short intro, itemize, explanation.}
Hereby we offer an open-source solution in the belief that it will help further development in this scientific field.
The task itself is by no means new, and as will be detailed in the following section, there are many software libraries that solve it.
Unfortunately, it is rare that such a library would be released with open source code, and these few are usually incomplete or outdated.

The purpose of the software presented here is also to serve as a benchmark rig for face tracking and eye tracking methods.
In fact, we evaluate many popular approaches to these problems and our implementation of the algorithms can easily be reused elsewhere.
Of special interest is our homography-based face tracker and our eye tracker based on iris radial symmetry.
To the author's knowledge, these two methods are entirely novel.
\todo{mention the testing data}
\todo{talk about available data sets}
\todo{what about dataset http://csr.bu.edu/headtracking/uniform-light/}

Where possible, we try to employ complex and rigid models that are specific to the tasks at hand.
There are several reasons.
In order to learn a generic machine learning algorithm, we would need much training data.
Unfortunately, none of the publicly available datasets for face and gaze tracking are suitable for our problem setting---most of them are designed for recognition from a static image, and some even lack color information.
Anyway, there has been much progress in the field of machine learning currently, and we can hardly compete with the manpower and computational capacity of these scientific teams.
Configuring a machine learning tool is simply not the aim of this thesis, by the author's personal preference.
