\chapter{Goals}

Given the fact that a majority of laptop computers are capable of constrantly recording the users using a video camera, it is attractive to analyse this stream.
We set off to the following goals:
\begin{itemize}

\item Build an \textbf{open-source} solution for interactive gaze tracking.
Gaze tracking is by no means a new task, and has been detailed in the previous section, there are many software libraries that solve it.
Unfortunately, it is rare that such a library would be released with open source code, and these few are usually incomplete or outdated.
We believe that sharing the source code will help further development in this scientific field.

\item Create a \textbf{benchmark rig} for evaluation of the tools in terms of performance.
By implementing several different methods for each of the subtasks at hand, we intend to compare them in terms of performance.
Further evaluation can help choose an appropriate method for a specific application.

Because our problem setting is already quite specific, we also provide testing data sets that fit the given conditions.
In order to properly evaluate the reasons why some methods perform better than others, it is necessary to make extra annotations about each testing sample.
Publicly available data sets lack this kind of information.

\item We decided to design a \textbf{novel algorithm} for face tracking, and another one for eye tracking.
Given the amount of algorithms that have been already presented in the literature, this plan may seem pointless.
However, it seems that there is still room for improvement, and discovering new methods may bring more insight.

\end{itemize}

Where possible, we try to employ complex and rigid models that are specific to the tasks at hand.
There are two reasons for this.
First and foremost, we would need too much training data in order to learn a generic machine learning algorithm.
It turns out that none of the publicly available datasets for face and gaze tracking are well suitable for our problem setting.
For example, most of them are designed for recognition from a static image, and some even lack color information.

Secondly, there has been much progress in the field of machine learning currently, and we can hardly compete with the manpower and computational capacity of these scientific teams.
Configuring a machine learning tool is simply not the aim of this thesis, by the author's personal preference.
