\chapter*{Conclusion}
\addcontentsline{toc}{chapter}{Conclusion}

We have designed and implemented a gaze tracker that can serve many real-world purposes.
The program has been only tested on for Linux but it does not depend on any specific features of the system and thus should be highly portable.

On average, we failed to attain a sub-pixel precision in eye tracking, as was the original goal.
Evaluation on real-world data shows that  the program is indeed capable of such a precision when supplied with flawless data, and it quite often succeeds even in more challenging cases.
It is mainly the outliers that spoil the average performance of eye tracking.

In order to deal with these outliers, we have designed a robust calibration method.

In terms of computational speed, the tracker provides an interactive performance of several frames per second on a typical computer.
The software is modular and highly adjustable, i.e., it can be configured for high precision or for decreased computational requirements at the expense of tracking quality.

The performance deteriorates steadily as the conditions become more challenging.
We have evaluated many test cases in detail and based on these, we provide an analysis of the actual necessary conditions for a smooth operation.

\section*{Future work}
\addcontentsline{toc}{section}{Future work}

A great challenge awaits our algorithm in that they should be compared to Artificial Neural Networks in terms of performance.
To this end, an existing software library such as \cite{deepgaze} shall be used.

So far, we did not manage to implement the resetting scheme for face trackers as it was described in Section \ref{s:impl-markers}.
This method is important for practical use of the program.

Following quite a similar thought, we should combine both of the eye trackers in a robust manner based on their tracking score.
The extreme case is to ignore one of the eyes completely if the tracking fails there.

Finally, the user comfort will be greatly improved by distributing some basic calibration data alongside the code, so that the program can be immediately tested upon download.

The interested reader is advised to download the most recent version of this program, instead of relying on the accompanying medium.
The source code, including some extra tools, is published at \ahref{https://github.com/addam/lokeye}.
