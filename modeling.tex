\chapter{Modeling}

In this section, design decisions behind all the necessary concepts are presented.
Many of them are deliberate, but all are based on solid theory or measurements.

\section{Eye}
\subsection{Eye Shape}

Throughout the literature, eye is modeled either as a sphere, or with an extra spherical section for the cornea.
Some sources also model the eyelids.

Sclera is white and pupil is black.
Cornea is something in between.

\subsection{Eye Movement}

Eye can rotate in all three degrees of freedom, in fact.

Donder's Law is important here because of the view axis offset from the optical axis.
Theoretically, position of the pupil might on its own not be sufficient to calculate gaze precisely.
That is not the case.

\section{Face}
\subsection{Face Appearance}
\subsection{Face Movement}
Human face consists of many muscles.
Like the pupil, they mainly serve for communication.
Of special interest for us are regions that remain mostly fixed to the skull in normal conditions.

\section{Gaze}

Under the assumption of small angle divergence, we can model movement of the pupil as simple translation.

Given the eye--face offset, onscreen gaze center is assumed to be a homography.
This saves us from explicitly modeling the viewed scene.
The non-linear factor is quite huge, unfortunately.

\section{Image}
The image as acquired from the camera is assumed to be a rectangular grid of colored points.
However, certain parts of the computation require a continuous image model in order to obtain sub-pixel precision.
This problem has three solutions in general:

\begin{itemize}
\item Use simple interpolation and ignore the inaccuracy induced.
This is the approach applied in this thesis.

\item Use simple interpolation on a blurred image so that the inaccuracy disappears.
This is the solution implicitly used by many software libraries.

\item Interpolate using a sophisticated function.
Surprisingly enough, commonly used interpolation functions are not suitable for a precise model.

\end{itemize}

