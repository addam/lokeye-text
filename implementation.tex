\chapter{Implementation}

Let us now use the theory and knowledge from previous two chapters to design a gaze tracker.
After sketching out the basic traits of the program, we continue by listing out various approaches to each of the subproblems.
There are indeed many options which make difference in computational difficulty, precision and robustness to possible difficulties in the input data.
They are presented here, and will be evaluated in the next section under varying conditions.

\section{Schema of Operation}

On a first run, the program needs a calibration step to learn specific parameters related to the user.
Some calibration is also necessary on each run and during longer periods of execution because of changing light conditions.

The pipeline of computation is presented in Drawing @ref.
Firstly, the face is locally tracked from the previous frame.
Only if this step fails, the program starts a global face localization procedure.
This first step results in a 3-dimensional translation and rotation mapping from the initial pose.

Next, the precise position of both eyes is established within a reasonable region in the face.
This steps requires the face to be localized properly in the image but since the eyes can move very quickly, this uses no prior from the previous frame.

Finally, the 6 degrees of freedom of the head and the 4 degrees of freedom of the eyes are fed into a gaze estimator that outputs coordinates in screen reference frame.

\section{Calibration}

The specific parameters being calculated during the calibration step depend on the particular methods being used for the remaining subtasks.
However, the calibration session always remains the same from user's point of view, so that is what will be described here.

\section{Face Tracking}

Our program supports two methods of face tracking, which represent two major groups of tracking techniques in general.
First is feature-based, second is appearance-based.

\subsection{Feature Tracking}
Along the lines of (@cite wang shi 16), we adopted a local feature-based tracking system.

\subsection{Active Appearance Models}

A surface within the face is tracked pixel per pixel.
Depending on the transformation allowed, we get different methods:

\begin{itemize}
\item Similarity transformation. 4 DoF. Aligning in grid makes no sense at all.
\item Affinity. 6 DoF. Aligning in grid makes little sense.
\item Barycentric transformation. 6 DoF. Possible to align in grid.
\item Perspectivity. 8 DoF. Possible to align in grid. Appropriate from a theoretical standpoint.
\end{itemize}

\section{Eye Tracking}

There are a plenty of eye trackers in our program.
From a broader perspective, we took four different approaches:
\begin{itemize}
\item Tracking the limbus. There should be a circle with a strong gradient directed outwards.
\item Tracking the iris. It should be radially symmetric with a strong gradient on the edges.
\item Segmentation. Skin, sclera, iris and pupil should each be uniform, but mutually different in color.
\item Machine learning. Instead of modeling, we can feed the image into a regression engine.
\end{itemize}

These approaches have been combined in many different ways, resulting in the following methods available to the user:

\subsection{Limbus gradient}
This method relies on the edge between the iris and the sclera.
Assuming that the iris is much darker than the sclera, their shared boundary (the limbus) should be displayed as a circle of high contrast.
We assume the eyelids to be of uniform color, so they have zero value in the gradient domain and can be neglected.

\subsection{Limbus polynomial gradient}
The eye region of the face is actually not uniform as it contains many wrinkles and directional light may cause strong shadow.
These facts cause a low-magnitude gradient that might modify the global optimum of the method described above.
In order to gain some robustness, we can apply a heuristic filter to the gradient values.
If designed with care, the method can still operate at nearly the same speed.

\subsection{Hough transformation}
We can use a voting scheme to find the limbus center.
This allows us to apply a sophisticated heuristic filter at almost no cost.
However, there is a free parameter: the resolution of the voting grid.
Using each pixel as a voting bin is usually the best choice, but it may fail on blurry images of high resolution because the votes will be distributed quite randomly over a wide region around the true center.

\subsection{Dark iris correlation}
Assuming that the iris is a dark disc and the rest of the image is much brighter, we can use basic template matching techniques to locate it.
An appropriately fast and reliable one is the normalized cross-correlation, using a slightly blurred black circle as the template.

This method appears to work especially well when tested in regions where the majority of people are brown-eyed and with white skin.
It can also easily get confused by dark spots around the eyes, such as glass rims or strong makeup.
Finally, this method requires most of the iris to be visible, which need not be true because of the viewing angle and the user's personal habit.

\subsection{Personalized iris correlation}
This is yet to be coded and tested.
The assumption of dark iris is wrong: the iris is often much brighter than the surrounding skin, especially in the case of blue-eyed people.
We can, however, rely on the 

\subsection{Iris radial symmetry}
We can define a model with few free parameters, relying only on the radial symmetry around the iris.
Being highly nonlinear, it is also difficult to optimize and may be trapped in a local minimum.

\subsection{Skin segmentation}
This is yet to be coded and tested.
It should be a switch that just disables processing of skin regions.

\subsection{Random forest regression}
This is yet to be coded and tested, but it is not going to work anyway.

\subsection{Convolutional neural networks}
This is yet to be coded and tested.

\subsection{Combined estimator}

Choosing several trackers that are sensitive to different deteriorations of the image, we can trade off some computational time for much robustness and precision.
Each of the trackers chosen should obviously fail in different conditions so that a majority of them is correct on every image.
Reasonable selection of trackers for a combined estimator is discussed in the next chapter.

\section{Gaze Estimation}

Computer screen is a plane in space, with pixels aligned in a regular square grid.
If the face and the eyes were also planar objects, then the gaze in screen coordinates would be related to the face and eyes position by a projectivity.
This is obviously not the case, and the face transformation has many more degrees of freedom than a planar rigid body would.
 
In order to properly model the mapping from our face and eye parameter space to two-dimensional gaze, we would have to actually estimate the three-dimensional model of the face.
Many programs have been published that follow this path (@cite), but we decided for a simpler approach.
Modeling the face would either require us to provide a strong a priori model, or it would require a longer calibration phase.

We decided to actually assume the gaze to be given by a projective transformation of the face and eye parameters.
This approach is limited and does never perfectly fit the data, but it is also very robust against overfitting. 
Only one method is available since both theory and measurements show that it works reliably and well.
It is a perspectivity.
\todo{\dots}
