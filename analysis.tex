\chapter{Problem Analysis}

This chapter provides a thorough analysis of the data stream provided by the camera, and of the objects displayed therein.
We discuss the anatomy of human face and eyes in particular, and all relevant technical details about commonly used web cameras.

\section{Conditions}

We expect that the user is working with their personal computer or a similar device such as a tablet.
The user and the device can freely move around; the only constraint there is that the camera must be fixed to the computer screen.
Throughout the calculation, we rely on that the user's face and both eyes are visible, and that the angle difference of the gaze and the camera direction is small.
We do not impose any further constraints on the face pose relatively to the camera.
For example, it is not necessary that the camera be upright, since such a constraint might be difficult for users working with a vertically rotated screen.

For a good calibration, we require active cooperation from the user: they are asked to watch a dot moving around the screen.
If necessary, this requirement can be somewhat lifted by using a calibration file defined earlier or by some completely different means of calibration.
We allow the user to move their head but we cannot \textit{rely} on this because many users (e.g. disabled ones) may have difficulties to move.

Ambient lighting must be good enough for the camera to deliver a high-contrast and sharp image.
The image brightness, white balance etc. are allowed to change abruptly and always.
There is, however, a strong requirement on the light being rather diffuse than directional, and making no sharp shadows on the user's face.
If the light conditions substantially start to differ from the ones during calibration, it may be necessary to re-calibrate.

\section{Human head}

\subsection{Face Anatomy}
The face is the frontal part of human head.
Facial muscles are specific in that they are attached to the skin, and their sole purpose is communication.
As seen in Figure \todo{@ref}, the mimic muscles cover almost all of the face.
Although they can be controlled by will, they often actuate subconsciously.
There are kinds of subtle movements, usually induced by basic emotions such as fear, that can not be prevented by will.
Generally, the facial skin is flexible with many degrees of freedom and the motion can affect almost all regions.

Of special interest for us are regions that remain mostly fixed to the skull in normal conditions.
These include most notably the chin, base of the nose and small outer regions around and slightly below the eyes.
The chin is especially prominent and easy to track, but may become completely misleading if the user opens their jaw.
\todo{\dots}

\todo{describe stuff like eyebrows, eyelashes, eyelids}

\subsection{Eye Anatomy}
\label{s.eyeanatomy}

The eyeball consists of several parts, most notably:
\begin{itemize}
\item Transparent \textit{cornea}, which makes a fixed lens and provides some mechanical protection
\item Flexible and reflective \textit{iris}, which serves as a diaphragm
\item Flexible \textit{lens} stretched by several muscles to control its optical magnitude
\item Purely white \textit{sclera}, which serves as a hard shell of the eyeball
\item \textit{Vitreous body} consisting mostly of a transparent gel to maintain the inner pressure
\item Light-sensitive \textit{retina}
\end{itemize}

Furthermore, each eyeball is embedded in a hole called the \textit{orbit} that provides fixation and actuation.
There are six separately controlled muscles stretching between to the eyeball and the orbit.

Throughout the literature, eye is modeled either as a sphere \cite{zhang13}, or with an extra spherical section for the cornea \cite{villanueva08}.
Some sources also explicitly model the eyelids.

\todo{rewrite this in a more educated manner}
Sclera is white and pupil is black.
Iris is something in between.
The iris diameter remains constant but the pupil diameter can change dramatically.

The inner and outer boundary of the human iris (the pupil and the limbus, respectively) are concentric circles.
Generally, the inner shape of the iris is less reliable of these two.
The pupil shape can also be permanently distorted or non-circular due to a damage or a disease \cite{bowyer16}.
In a healthy case, its radius changes depending on the amount of incoming light and the emotional state of the user.
The constriction is controlled by the \textit{parasympathicus}, a nervous system generally related to comfortable actions such as eating and sleeping.
The dilation is controlled by the \textit{sympathicus}, which in turn is the main actuator of the so-called \textit{fight or flight} stress response.
Both sympathicus and parasympathicus are parts of the \textit{autonomous nervous system} and therefore cannot be directly controlled by will.

In case of a disease or an injury, it is also possible that some of the muscles stop working properly and the inner iris shape is no longer a circle.\todo{image}

\subsection{Eye Movement}

Human eyes have developed not only as an organ of sight, but also as a means of communication.
When attending to an object or another person, people will usually turn their eyes directly towards it.

The main reason of this is that the overall acuity of our visual system increases towards an area near the optical axis.
The corresponding spot on the retina is called the \textit{fovea}, and it is located about $5\degree$ horizontally towards the nose \cite{villanueva08}.
It covers only about $1.5\degree$ of the visual field.
The density of photopic (daylight-sensitive) cells in the fovea is up to 20 times more than in the peripheral areas of the retina.

Although gaze can be precisely controlled by will, there are many peculiarities about eye motion that depend on old brain circuitry and are common to all humans.
Assuming that each muscle can apply force only by stretching, and not extending itself, the six muscles provide three degrees of freedom to the eye motion---but only two DoF are necessary to control the gaze direction.

A third DoF, namely rotation around the optical axis (the \textit{roll}) is actively used by the brain.
The effect is slight, but because fovea is offset from the optical axis, this could make the gaze direction unpredictable given the optical axis only.
Fortunately, the roll is deterministic with respect to the remaining two rotation angles.
This fact is known as the \textit{Donder's Law} \cite{hansen10}.
Thanks to it, the effect of eye roll can be implicitly precalculated during calibration.

\todo{saccades, fixation, smooth pursuit}

\section{Gaze}
\label{s:gaze-model}

The gaze is the direction of focus of the user.
We assume that it is pointed to somewhere on the computer screen.

Gaze is affected by both the head pose and eye rotation.
Although it may seem more efficient to only move our eyes when using a computer, people usually move their heads.
In fact, prolonged static pose of the head can cause pain in the spine.

\todo{geometric description of the scene}
\todo{accompanied by a figure}

Under the assumption of small angle divergence, we can model movement of the pupil as a simple translation.
It is important to note that because the head and the eyes differ by an order of magnitude in scale, gaze estimation is numerically unstable.

We avoid explicit modeling of the viewed scene by assuming that the gaze is given by a homography, as it will be detailed in Section \todo{@ref}.
The non-linear factor is quite huge, unfortunately.

\section{Camera}

A suitable yet very simple mathematical model of a video camera is the pinhole camera.
Light rays passing through a certain point in space (figuratively, the pinhole) are projected onto an image plane.
The axis of symmetry of this system is called the optical axis, and the intersection of the optical axis with the image plane is assumed to be the origin point in image coordinates.

Real-world cameras can suffer from many kinds of degeneracies off this model:
\begin{itemize}

\item
Imprecise manufacturing.
The image origin point may be offset from the optical axis.
The sensor may be stretched and skewed, so that orthonormal vectors in image plane are not always sensed as orthonormal.

\item
Blur.
Refractive lens are usually inserted into the optical path so that light need not pass through an infinitely small pinhole, but rather though a small disc.
This approach results in that only light rays originating from a certain surface in the scene, known as the focal plane, are properly projected onto the image plane.
Points from a plane parallel with the focal plane will be displayed as if convolved with a disc kernel; the radius increases with the parallel distance from the focal plane, and the shape is mostly a projection of the camera iris.

The lens may be dirty or scratched, which casues a slight and uniform foggy blur.
In very small cameras with relatively high resolution, the wave nature of light may cause additional blur by diffraction at edges of the camera iris.

\item
Lens aberration.
The lens itself is a thick solid object and therefore can never fulfill its physical model perfectly.
The nonlinear effect usually manifests itself by stretching sensed points in or out relatively to the image origin point.
If carefully measured, this geometric transformation within the image plane can be very well cancelled.

Because the refractive index of materials varies with wavelength (this fact is known as dispersion), the nonlinearities are also wavelength dependent.
There are software tools to reduce percieved color aberration, but this problem is under-determined and can never be solved exactly.
A proper solution would require to densely sample the spectrum at each image point but we only have three values roughly corresponding to red, green and blue.

These effects are often well compensated in high-end cameras by sophistication of the optical system.
On the other hand, they also decrease with the lens size, and cameras with a narrow or almost closed iris can be fairly well aproximated as pinhole cameras with no lens.

\item
Moiré.
In most consumer cameras, image colors are obtained by attaching different color filters to each cell of the light sensor.
\todo{explain some more}
Contrast on subpixel level makes each color channel sum up to a different amount, even if there are no colored objects in the scene.
When imaging fine black-and-white colored structures, spurious and highly saturated colors may appear.

This effect can hardly occur when displaying human faces.
Quite to the contrary, it can be used for manual focusing of the camera, if necessary.
Moiré is an optical effect between the imaged object and the light sensor, and usually will not be affected by image processing such as denoising algorithms.
We can put a black and white grid nearby the user's head and tune the camera focus until color moiré appears.

\end{itemize}

\section{Image}
\label{s.imagemodel}

The image acquired from the camera is a rectangular grid of colored points, expressed as a matrix $\mat M \in \R^{n\times m}$.
However, certain parts of the computation require a continuous and smooth image model in order to obtain a sub-pixel precision.
In such cases, the image function is defined as
$$I(\vec p) = \todo{formula}.$$

We need to geometrically transform (i.e., stretch) the image, and evaluate it at random positions.
In general, this problem has three possible solutions.
Firstly, we can use simple interpolation and ignore the inaccuracy induced.
This is the approach applied in this thesis.

The second option is to use simple interpolation on a blurred image in the hope that the inaccuracy disappears.
This solution is implicitly used by many software libraries.

Finally, it is possible to interpolate using a sophisticated function.
We need to calculate image derivatives up to 2nd order, so , such as
\todo{mention Lanczos function $\mathrm{sinc}(x) \cdot \mathrm{sinc}(x/2)$}
\todo{mention $\frac 1 2 (\cos(x) + 1)$ is symmetric, just somewhat too sharp}

However, many commonly used interpolation functions are not suitable for a precise model.
An interpolator should obviously not be biased towards zero, i.e., interpolation of a constant image should be a constant.
Because of this, functions such as $\mathrm{sinc}$ can only serve as weights in a weighted average.
The vector formula  often makes image derivatives prohibitively hard to calculate.
In the end, this may lead us to only estimating the image derivatives by a simple formula---but by making such a step we effectively classify to the first category above.

\subsection{Image derivatives}
The partial derivatives of an image are estimated as
$$\todo{formula}.$$

Note that each derivative is sampled at a slightly shifted position.
While this may seem needlessly picky, precise coordinates become very important when working with downscaled images (e.g., in Section \todo{@ref pyramid}).
A half-pixel shift in a downscaled image can represent a very large distance in the original pixel grid, and ignoring this displacement would actually make our algorithms fail.
