\chapter{Algorithms}

This chapter provides mathematical tools and computational methods for the problems to be presented in the next section.

\section{Linear algebra}

\subsection{Singular Value Decomposition}
\textit{Input:} matrix $\matA$\\
\textit{Output:} orthonormal matrices $\mat U, \mat V$ and a vector $\mat s$ such that $\matA = \mat U \cdot \diag(\mat s) \cdot \mat V\T$.\\

\todo{Jacobi algorithm!}

The system matrix in our case is of shape $K \cdot (n - 1) \times m$, which leads to an overall time complexity of $O(m^2 K (n - 1))$.

\subsection{Linear Solving}
\textit{Input:} matrix $\matA$, vector $\mat b$\\
\textit{Output:} vector $\matx$ such that $|\matA \matx - \mat b|$ is minimal.\\

\todo{\dots}

\subsection{Homogeneous Linear Solving}
\textit{Input:} matrix $\matA$\\
\textit{Output:} vector $\matx$ such that $|\matx| = 1$ and $|\matA \matx|$ is minimal.\\

Let us consider the SVD\todo{abbreviation} of $\matA = \mat U \cdot \diag(\mat s) \cdot \mat V\T$.
The result $\matx$ is the $i$-th column of $\mat V \T$, where $i$ is chosen so that $\mat s_i^2$ is minimal.

\begin{proof}
Let us define a vector $\mat v = \mat V \matx$.
This is effectively a change of basis, and because $\mat V$ is an orthonormal matrix, the norm $|\mat v| = |\mat x|$ is preserved.
Because also $\mat U$ is an orthonormal matrix, it can be safely crossed out of the norm $|\matA \matx| = |\diag(\mat s) \cdot \mat v|$.
This in turn can be expressed as the square root of $\sum_i (\mat s_i \mat v_i)^2 = \sum_i \mat s_i^2 \mat v_i^2$.

Finally, this constrained minimization problem can be solved by the method of Lagrange multipliers.
Equating $\nabla(\sum_i \mat s_i^2 \mat v_i^2) = \lambda \nabla (\sum_i \mat v_i^2)$ leads to the system of equations $\mat s_i^2 \mat v_i = \lambda \mat v_i$ for all $i$.
These necessary constraints are satisfied only by choosing $\lambda = \mat s_i^2$ for some $i$.
Deliberately assuming that all $\mat s_j^2 \neq \mat s_i^2$ for all $j \neq i$, we must also set the remaining elements $\mat v_j = 0$, and therefore $\mat v_i = 1$.
In order to find a global minimum, we pick $i$ such that $\mat s_i^2 \mat v_i^2 = \mat s_i^2$ is minimal.
The result is $\mat x = \mat V \inv \mat v = \mat V \T \mat v$.
\end{proof}

If $\mat s_i^2 = \mat s_j^2$ for some $j$, the global minimum is not unique but this method finds a correct solution anyway.
This can be proven by adding a small perturbation to $\mat s_j$.

\subsection{Quadratic polynomial fitting}
\textit{Input:} set of points $P = \{\mat p^i \in \R^2 \}$ and a function $\textrm{score}: \R^2 \rightarrow \R$.\\
\textit{Output:} quadratic polynomial $Q \in \R^2 \rightarrow \R$ fitting $Q$ in least-squares sense and its maximum $\mat x$.\\

Given several samples of a real function nearby its maximum, we would like to get a precise estimate of the maximum location.
Although this topic belongs to real analysis, it is listed here because most of the actual computations involved belong to linear algebra.

Let us express the least-squares fit as a linear system:
$$(x^2, xy, x, y^2, y, 1)\T \mat q = \textrm{score}(\mat p^i) \text{ for each } i, \text{ where } \mat p^i = (x, y)\T.$$
The polynomial $Q$ can be expressed in terms of the solution $\mat q$ as the symmetrical matrix
$$\mat Q = \begin{pmatrix}
 2 \mat q_1 & \mat q_2 & \mat q_3 \\
 \mat q_2 & 2 \mat q_4 & \mat q_5 \\
 \mat q_3 & \mat q_5 & 2 \mat q_6
\end{pmatrix}.$$
Let us split this matrix into the following blocks:
$$\mat A = \begin{pmatrix}
 2 \mat q_1 & \mat q_2 \\
 \mat q_2 & 2 \mat q_4 \\
\end{pmatrix},
\mat b = \begin{pmatrix}
 \mat q_3 \\
 \mat q_5 \\
\end{pmatrix}.$$
The global extremum $\mat x$ of this polynomial satisfies $\mat A \mat x = -\mat b$.

\begin{proof}
\todo{\dots}
\end{proof}

\subsection{Principal Component Analysis}
\todo{is this used in the program?}

\section{Nonlinear optimization}

\subsection{Gradient descent}

\todo {line search}

\section{Fourier analysis}

\subsection{Fast Fourier Transform}
\textit{Input:} vector $\mat v \in \C^n$.\\
\textit{Output:} vector $\mat f \in \C^n$ such that $\mat x_i = \sum_j \mat f_j \omega^{i j}$ for all $i$, where $\omega$ is $n$-th root of unity.\\

Let us define the function $\mathrm{FFT}: \R^{2m} \rightarrow \R^m$.

\begin{proof}
@cite CLRS
\end{proof}

\subsection{Convolution}

\textit{Input:} matrices $\mat M \in \R^{n \times m}, \mat T \in \R^{k \times l}$.\\
\textit{Output:} matrix $\mat R \in \R^{|n-k| \times |m - l|}$ as defined below.\\

\subsection{Normalized Cross-Correlation}

\textit{Input:} matrices $\mat M \in \R^{n \times m}, \mat T \in \R^{k \times l}$.\\
\textit{Output:} matrix $\mat R \in \R^{|n-k| \times |m - l|}$ as defined below.\\

\todo {there are more use cases beyond the obvious one.}

\section{Geometry}

\subsection{Direct Linear Transformation}

\textit{Input:} set of point pairs $\matx^k \leftrightarrow \matp^k$ related by an unknown homography.\\
\textit{Output:} homography matrix $\matH$.\\

There are, loosely speaking, two difficulties in homography fitting as compared to affine transformations.
Firstly, it has a nonlinear component so an algebraic least-squares solution is generally different from a geometric least-squares solution.
The former can be obtained in a much more stable way than the latter, so we start with algebraic least squares fitting and use the result for initialization of a nonlinear optimization method.
Secondly, the measured data points are defined only up to scale, so it is not possible to formulate a linear system straight away.
A technique called Direct Linear Transformation is widely used to overcome this issue.

We are presented with a set projective correspondences of the form
$$\matH \matx^k = \alpha_k \matp^k, 1 \leq k \leq K,$$
where $\alpha_k$ are unknown scalars and $\matH$ is an unknown matrix that we want to calculate.

Unfortunately, these equations do not form a linear system with respect to the unknown matrix $\matH$ because there is an unknown scale factor $\alpha_k$ involved in each equation.
Differences among the scale factors make up the perspective part of the homography, so although they can make the problem numerically unstable, we cannot impose any limits on these values.
The core idea of Direct Linear Transformation is to find a set of vectors that must be orthogonal to the solution, and then solve the resulting homogeneous system.

Let us define a set of antisymmetric matrices $\{\matM^{i, j} \in \R^{n \times n}\}$, $1 \leq i < j \leq n$, each having only two nonzero elements:
$$\matM^{i, j}_{i, j} = 1,
\matM^{i, j}_{j, i} = -1.$$
From this definition it follows that for any $\matp \in \R^n$, the product $\matp\T \matM^{i, j} \matp = \matp_{i} \matp_{j} - \matp_{j} \matp_{i} = 0$.
The vectors $\left\{ \matM^{i, j} \matp \right\}, 1 \leq i < j \leq n$, are all orthogonal to a given $\matp \in \R^n$.

With increasing dimension, this set of vectors becomes heavily redundant.
Clearly, a minimal solution would form a basis of the subspace $\R^n / \matp$ orthogonal to $\matp$ and thus would consist of $n - 1$ vectors, whereas this approach generates $\frac {1} {2} n \cdot (n - 1)$ vectors.
Some sources\cite{MVG} suggest to pick an arbitrary subset of size $n - 1$ from the matrices defined above and use these for the whole data set.
A more proper solution is to select these matrices specifically for each correspondence pair $\matx^k \leftrightarrow \matp^k$ so as to avoid possible degeneracies.
In particular, we find the largest vector element $\matp_l$ (in absolute value) and then select all matrices $\matM^{i, j}$ such that $i = l$ or $j = l$.
Each of the vectors generated this way contains the value of $\matp_l$ at a different position (i.e., a different vector element), therefore the vectors are linear independent and they form a basis of $\R^n / \matp$.

Now we can formulate the linear system.
It will consist of $K \cdot (n - 1)$ equations, where $K$ is the number of correspondence pairs $\matx^k \leftrightarrow \matp^k$.
Its unknowns will be precisely the elements of the matrix $\matH$.
Each equation of the system represents a constraint of the form $\matp\T \matM \matH \matx = 0$, which is the orthogonality constraint as proposed above.
This constraint can be reformulated in terms of the Frobenius inner product $\langle \matA, \matB \rangle_F = \sum_{i, j} \matA_{i, j} \cdot \matB_{i, j}$ and the vector outer product as follows:
$$\matp\T \matM \matH \matx = \sum_{i, j} (\matp\T \matM)_i \matH_{i, j} \matx_j = \langle \matp\T \matM \matx\T, \matH \rangle_F = 0.$$
Viewing the matrices as vectors of their elements, this finally leads to an equation in the suitable form:
$$\vec (\matp\T \matM \matx\T)\T \cdot \vec(\matH) = 0,$$
where $\vec : \R^{m \times n} \rightarrow \R^{mn}$ stacks all matrix elements to a vector, in row-major order.
Each correspondence pair and each selected antisymmetric matrix $\matM^{i, j}$ for that pair provide one such equation, and the row vectors $\vec ({\matp^k}\T \matM^{i, j} {\matx^k}\T)\T$ can be stacked to form the system matrix.

Homogeneous linear systems such as this one can be solved using the Singular Value Decomposition (SVD).

\todo{\dots}
However, if the input vectors are degenerate and only span a subspace of $\R^m$, there are multiple solutions and not all of them are proper solutions of the original equation $\matH \matx^k = \alpha_k \matp^k$.
In particular, it is possible that a solution will produce points at infinity, meaning that $\matp_n \approx 0$ and that the Cartesian counterpart of $\matp$ is undefined.
Instead of avoiding such degeneracies explicitly, we loop through the resultant singular vectors and choose the one with the minimal reprojection error.

\todo{normalization: subtract mean, enforce unity variance}

\subsection{Four-point homography}

A minimal case of a 2-dimensional homography estimation is a correspondence of 4 point pairs.
The general formula as generated by DLT\todo{abbreviation?} would be needlessly bloated for this setting.

Let us write the point sets as columns of a $3 \times 4$ matrix, and let us denote the equivalence of two point sets up to scale by the $\sim$ (tilde) character:
$$\matA \sim \matB \Leftrightarrow \exists \alpha_1 \dots \alpha_4 : \matA \cdot \begin{pmatrix}
 \alpha_1 & & \\
  & \ddots & \\
 & & \alpha_4
 \end{pmatrix} = \matB.$$
Also, let us declare the following set as the \textit{canonical configuration}:
$$\matC = \begin{pmatrix}
 1 & 0 & 0 & 1 \\
 0 & 1 & 0 & 1 \\
 0 & 0 & 1 & 1
 \end{pmatrix}.$$
We will now present the formulas to convert an arbitrary 4-point set to and from the canonical configuration.

For a given point set $\matA = \left( \mat{a}, \mat{b}, \mat{c}, \mat{d} \right)$, the homography $\mathrm{can}(\matA)$ is defined as:
$$\mathrm{can}(\matA) = \begin{pmatrix}
 \alpha_1 (\mat{b} \times \mat{c}) \T \\
 \alpha_2 (\mat{c} \times \mat{a}) \T \\
 \alpha_3 (\mat{a} \times \mat{b}) \T \\
 \end{pmatrix},
$$
where $\alpha_1 \dots \alpha_3$ are chosen so that $\mathrm{can}(\matA) \cdot \mat{d} = \left( 1, 1, 1 \right)\T$.
Note that this requires no linear solving, just an element-wise division.
If the points $\matA$ are affine independent, it clearly follows that $\mathrm{can}(\matA) \cdot \matA \sim \matC$.

The homography $\mathrm{dec}(\matA)$ is, in turn, defined as:
$$\mathrm{dec}(\matA) = \left( \alpha_1 \mat{a}, \alpha_2 \mat{b}, \alpha_3 \mat{c} \right),$$
where $\alpha_1 \dots \alpha_3$ are chosen so that $\mathrm{dec}(\matA) \cdot \left( 1, 1, 1 \right)\T = \mat{d}$.
This requires us to solve the $3 \times 3$ linear system $\left(\mat{a}, \mat{b}, \mat{c} \right) \cdot \left(\alpha_1, \alpha_2, \alpha_3 \right) \T = \mat{d}$.
As a result it follows that $\mathrm{dec}(\matA) \cdot \matC \sim \matA$.
Note that the vector $\mat{d}$ is exactly the right-hand side of this system.
A neat side-effect is that the inverse matrix $\mathrm{dec}(\matA) \inv = \mathrm{can}(\matA)$ has correct scale. 
More importantly, it means that the mapping $\mathrm{dec}$ is linear with respect to all elements of $\mat{d}$.

These two mappings can be composed to provide any four-point homography as necessary:
$$\mathrm{dec}(\matB) \cdot \mathrm{can}(\matA) \cdot \matA \sim \matB.$$
The resulting mapping is linear with respect to the last column of $\matB$.

\todo{what are the overall computational requirements?}

\subsection{Derivatives of a homography}
\label{s.homderivatives}

We wish to calculate the derivative of a four-point homography with respect to the coordinates of its four control points.
The linearity of $\mathrm{dec}(\matA)$ implies that the derivative $\frac{\partial} {\partial \mat{d}_i} \mathrm{dec} \left( \mat{a}, \mat{b}, \mat{c}, \mat{d} \right) = \mathrm{dec} \left( \mat{a}, \mat{b}, \mat{c}, \mat{e}_i \right)$, where $\mat{e}_i$ is the unit vector for the coordinate axis $i$.

Furthermore, we need to calculate derivatives with respect to $\mat a$, $\mat b$ and $\mat c$.
Theoretically, we could use the fact that any permutation of the canonical points can be expressed as a homography,\footnote{
The homography $\mat{R} = \begin{pmatrix}
 0 & 0 & 1 \\
 -1 & 0 & 1 \\
 0 & -1 & 1
 \end{pmatrix}$ has the effect $\mat{R} \cdot \matC \sim \begin{pmatrix}
 1 & 1 & 0 & 0 \\
 1 & 0 & 1 & 0 \\
 1 & 0 & 0 & 1
 \end{pmatrix}$.
} and incorporate these homographies into the formulas above.
Unfortunately, there are several unknown scale factors involved that make such an approach inefficient.
The preferable method is to explicitly permute the points so that the point in question becomes the last one, and to repeat the original scheme.

\todo{what are the overall computational requirements, per pixel?}

\section{Machine learning}

\todo{Ransac}
\todo{Boosted random forests}
