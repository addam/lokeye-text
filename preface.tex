\chapter{Introduction}

The main idea behind this program is to view human eye as a means of communication.
It is rather intuitive for us to recognize other people's gaze just by looking at their face thanks to high contrast coloring of the eye itself.
The corresponding brain circuitry develops early in infants (@cite) and is equivalent among individuals.
This seems to be an evidence of evolutionary purpose, that the human eye is built in a way so as to display the gaze clearly.
Therefore it should be possible for a computer to estimate gaze from the data us humans have available, that is, a color image.

Thanks to the fact that almost every laptop computer or cell phone is now equipped with a video camera, the hardware requirements of this software should be easy to satisfy.
If applications are built on top of this program, minimal hardware requirements like this allow them to be used almost immediately upon download and completely for free.

Eye trackers can actually be used for communication, namely for human-computer interaction.
Wide variety of algorithms have been proposed (@cite) for on-screen keyboard and general desktop control.
The user's center of attention can also serve as a cue for adaptive rendering in video games because the rest of the screen need not be displayed in full detail.
Possible applications in gaze tracking in cell phones are a vast topic, such as locking the screen when not looked at or displaying a note when somebody looks over your shoulder.

\section{History}

Perhaps a more appropriate heading of this section would be \textit{Unrelated Work}.
Indeed, before we get to the overview of our software competitors, we should shortly review past research of gaze tracking in general.
Its history spans half a century before the computer era.

The interest in eye tracking originates in the field of psychology.
Gaze designates the focus of attention of the subject, which in turn can tell much about ongoing cognitive processes.
Furthermore, eye movement itself is the result of a rather complex neural system; nowadays, it is perhaps the best studied part of human cognition.

In order to record data with reasonable precision, sophisticated mechanical structures were often built around the subject's head.
Eye movement was then measured either directly from an object attached to the eye, or indirectly from small displacements in the eye region.
An especially precise technique was developed by Yarbus in 1954 and requires to stick a mirror to the surface of the eye using a small suction cup.\footnote{The article is not cited here because it has been published only in Russian and does not seem generally available.}
Eye movement can then be recorded directly on a piece of photographic film.

A method known as \textit{electrooculography} provides a less invasive alternative.
There is a constant voltage gradient across the eye from back to front, in magnitude of about 1 mV.
Rotation of this small dipole is measurable, so it is possible to almost directly measure the speed at which the eye moves.
The advantage gained is temporal resolution: this method allows to draw a graph of angle against time.
However, the actual direction is an integral of the measured value, and therefore tends to deteriorate and spatial resolution is generally poor.

Details on other purely analog methods such as this can be found in a 1967 book by Yarbus \cite{yarbus1967}.
The various methods suggested for eye tracking since the end of 19th century can be seen as a story of human curiosity and a proof of the effort directed towards this area.

\section{Related Work}

Rapid development of computers and digital video cameras removed most of the hardware constraints mentioned so far.

