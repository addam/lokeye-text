\chapwithtoc{Attachments}

\setcounter{section}{0}
\stepcounter{chapter}
\renewcommand{\thesection}{\Alph{section}}

\section{Directory structure}
\label{s:dirstructure}

The attached data is organized as follows:

\begin{itemize}
\item The directory {\tt bin/} contains \dots

\item {\tt doc/install.html} and {\tt doc/usage.html} is the user documentation of the program.
The file {\tt doc/code.html} contains developer documentation and overall notes about the structure of the code.

\item {\tt extern/} contains the less common external libraries that are needed to compile and run the program.

\item {\tt src/} contains the whole source code of the program.
Apart from it, a Python script {\tt src/io\_export\_tracks.py} is provided that serves for exporting the camera calibration conveniently from Blender as input to our program.
Further files for testing of the source code are provided in {\tt src/test/}.

\item The directory {\tt data/} contains data for testing of the program.

\item Finally, the file {\tt thesis.pdf} is the electronic version of this document.

\end{itemize}

\section{Testing Data}

\todo{\dots}

\section{Library interface}

The software can be readily used as a gaze tracking library.
There are several decisions that the host application must make in order to initialize the tracking algorithms.
\todo{\dots}

The file {\tt src/example\_main.cpp} is a simple gaze tracking application and an example of the application interface.
There are more similar files prefixed with {\tt example\_} that showcase smaller bits of the code, and often use methods that should be hidden from the host application.
