\chapter{Results}

Having designed and implemented the gaze tracker, we shall now evaluate its performance.
There are three aspects to consider when talking of performance: robustness, precision and speed.
We compare the various methods against each other and against the state of the art.

\section{Face Tracking}
\todo{intuitively compare the face tracking methods on a real image}

\todo{compare them an a computer-rendered face image, if possible}

\section{Iris Tracking}

\subsection{Failure Factors}
\todo{correlation of success rate of each eye tracker to qualitative aspects of the images}
\todo{repeated for each of the three data sets, referenced to the attachment for their description}

\section{Gaze Estimation}
\todo{the overall success rate of the algorithm on several real videos}

The program does not perform all that well when compared to the state of the art, but it is certainly much more than a proof of concept.

The results on flawless video streams are already usable, and challenging situations could perhaps be solved with more coding manpower.
\todo{\dots}
