\chapter{Conclusion}
\addcontentsline{toc}{chapter}{Conclusion}

We have designed and implemented a gaze tracker that can serve many real-world purposes.

On average, we failed to attain a sub-pixel precision in eye tracking, as was the original goal.
Evaluation on real-world data shows that  the program is indeed capable of such a precision when supplied with flawless data, and it quite often succeeds even in more challenging cases.
It is mainly the outliers that spoil the average performance of eye tracking.

In order to deal with these outliers, we have designed a robust calibration method.

In terms of computational speed, the tracker provides an interactive performance of several frames per second on a typical computer.
The software is modular and highly adjustable, i.e., it can be configured for high precision or for decreased computational requirements at the expense of tracking quality.

The performance deteriorates steadily as the conditions become more challenging.
We have evaluated many test cases in detail and based on these, we provide an analysis of the actual necessary conditions for a smooth operation.

\section{Future work}

It would be nice to integrate some existing libraries \cite{deepgaze}.

We should provide calibration data to enable quick testing.

The performance of each face tracker should be evaluated and if it considerably deteriorates, it should be reset.

Out of the two eye trackers, we should choose the one with the greater fitting score.
